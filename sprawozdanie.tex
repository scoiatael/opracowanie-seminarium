\documentclass[11pt]{scrartcl}

\usepackage[OT4,plmath]{polski}
\usepackage{amsmath,amssymb,amsfonts,amsthm,mathtools}
\usepackage{color}
\usepackage{fontspec}
\usepackage{listings,times}
\usepackage[scaled]{beramono}
\newfontfamily{\lstsansserif}[Scale=.85]{Latin Modern Mono}

\definecolor{keywords}{RGB}{30,200,30}
\definecolor{comments}{RGB}{210,40,40}
\usepackage{listings,times}
\lstset{
  basicstyle=\lstsansserif,
  breaklines=true,
  showstringspaces=false,
  literate={fun}{$\lambda$}1,
  keywordstyle=\color{keywords},
  commentstyle=\color{comments}\emph}
\renewcommand\lstlistingname{Przykład}

\usepackage{bbm}
\usepackage[colorlinks=true, urlcolor=blue]{hyperref}
\usepackage{url}

\setmainfont{Latin Modern Roman}

\title{Jak (prawie) bezboleśnie napisać kompilator języka funkcyjnego?}
\subtitle{Opracowanie seminarium ,,Implementacja języków funkcyjnych''}
\author{Łukasz Czapliński}
\date{\today}

\begin{document}
\maketitle

\vfill
\section*{Streszczenie}
W pracy ,,Compiling with Continuations'' Andrew W. Appel opisał konstrukcję
kompilatora SML/NJ.\@ W trakcie seminarium ,,Implementacja języków funkcyjnych''
napisaliśmy prosty kompilator języka z rodziny ML.\@ W  tej pracy opisuję, które
idee okazały się szczególnie przydatne i na jakie problemy natrafiliśmy.

\vfill

\section{Wstęp}
Wielu programistów uważa, że pisanie kompilatorów jest ciężkie. Do tego stopnia,
że podstawowy podręcznik nosi potocznie nazwę ,,Książki o smokach'' (Dragonbook,
\cite{Dragonbook}).
Skoro tak ciężkie jest napisanie prostego kompilatora, to jak ciężkie może być
napisanie kompilatora języka wysokiego poziomu, z rodziny ML?\@ W ramach
seminarium “Implementacja języków funkcyjnych” w Instytucie Informatyki
Uniwersytetu Wrocławskiego postanowiliśmy taki kompilator napisać.

Za podręcznik przyjęliśmy ,,Compiling with continuations'' \cite{Appel}, opisujący
implementację języka SML.
Poniższa praca jest podsumowaniem tego, co udało nam się osiągnąć.
Kolejność rozdziałów odpowiada kolejności wykonywanych przez nas prac.
Nacisk położony jest na kroki kluczowe dla działania kompilatora.

\section{Architektura kompilatora (Języki pośrednie)}
Celem kompilatora jest transformacja tekstu napisanego przez użytkownika
na kod maszynowy, działający na danej architekturze i wykonujący program zadany
przez użytkownika.
Większość kompilatorów wyróżnia przynajmniej dwa stadia: parsowanie tekstu do
drzewa składniowego (\textit{Abstract Syntax Tree} -- AST) -- \textit{analiza}, a następnie
generowanie kodu maszynowego z AST -- \textit{synteza} \cite{Dragonbook}.
Im prostszy język w którym kompilator jest napisany, tym trudniej dodać więcej
stadiów.
Ponieważ nasz kompilator napisany został w Ocamlu (języku funkcyjnym wysokiego
poziomu \cite{Leroy16}), kompilacja miała kilka stadiów; składała się z parsowania tekstu do
AST, a następnie z ciągu transformacji, które upraszczały, optymalizowały oraz
tłumaczyły AST na kolejne języki pośrednie.
W finalnej wersji wyróżnić można następujące języki pośrednie:
\begin{itemize}
\item RawMiniML -- pierwszy język, powstały ze sparsowania tekstu użytkownika.
  Można powiedzieć, że ten właśnie język jest przez nas kompilowany.
\item MiniML -- powstały z poprzedniego przez wyprowadzenie typów
  (zaimplementowany został system inferencji typów) i usunięciu sugestii typów
  danych przez użytkownika na rzecz informacji o typie każdego węzła AST.
\item Lambda -- powstały przez usunięcie typów i wszystkich zależnych od nich konstrukcji z MiniML.
\item CPS -- Continuation-Passing Style -- język w którym będzie wykonana większość optymalizacji.
\item AbstractM -- asembler procesora na jaki będziemy kompilować.
\end{itemize}




\section{Opis kompilowanego języka}
MiniML jest uproszczonym językiem z rodziny ML, silnie wzorowanym na Ocamlu
\cite{Leroy16} oraz SML-u \cite[Chapter 1. What is ML?]{Appel}.
Podobnie jak one, jest statycznie typowany i posiada system inferencji typów.
Składnia jest podobna do innych języków z tej rodziny, z kilkoma wyjątkami:
\begin{lstlisting}
datatype 'a list = nil | cons of {'a, 'a list}
in

let rec rev_append ['a] (xs ys : 'a list) : 'a list =
  match xs with
  | nil    => ys
  | cons p => rev_append['a] (#1 p) (cons['a] {#0 p, ys})
  end
in fn ['a] (xs : 'a list) => rev_append['a] xs (nil['a])
\end{lstlisting}
\begin{itemize}
\item Deklaracje typów są widoczne tylko w określonym bloku kodu.
\item Nie ma ograniczenia w nazewnictwie typów danych i konstruktorów
  (inaczej niż w Ocamlu, podobnie jak w SML-u).
\item Pattern matching nie wprowadza nowych zmiennych -- do pól konstruktora
  trzeba się odwołać przez dekonstruktor: \lstinline|\#1 p| oznacza drugi element \lstinline|p|
  (gdzie \lstinline|p| jest stworzone przez konstruktor o arności przynajmniej 2).
\item Polimorficzne funkcje muszą deklarować parametry typowe przez \lstinline|['a]|.
\item Nie posiada systemu modułów.
\end{itemize}



\section{Optymalizacje typowe dla języka ML}
Pierwszym krokiem kompilacji na jaki się zdecydowaliśmy to transformacja do
prostego języka opartego na rachunku lambda.
Transformację tę można rozważać jako funkcję $F$, która jako argument przyjmuje
$\Gamma$- informację o typach wszystkich wyrażeń w programie, oraz $A_{ML}$-
wyrażenie w języku MiniML, a zwraca $A_L$- wyrażenie w języku Lambda.

Nasz język pośredni Lambda pozbawiony jest typów, tak więc na tym etapie
wszystkie konstrukcje, które w jakiś sposób potrzebują systemu typów muszą
zostać zamienione na prostsze, które nie potrzebują tej dodatkowej informacji.
W językach rodziny ML zwykle do nich należą \cite[Chapter 4: ML-specific optimizations]{Appel}:
\begin{itemize}
\item nazwane pola rekordów;
\item moduły, sygnatury, funktory;
\item typy danych, konstruktory;
\item pattern-matching;
\item wyjątki.
\end{itemize}
Dodatkowo uwagi wymagają:
\begin{itemize}
\item testowanie równości (to, jaki kod będzie potrzebny do porównania
  równości dwóch obiektów ściśle zależy od tego co wiemy o ich typie)
  \cite[Chapter 4.3: Equality]{Appel}.,
\item optymalizacja dostępu do pamięci (niektóre typy mogą pozwolić na lepszą alokację
  pewnych komórek pamięci) \cite[Chapter 4.4: Unboxed updates]{Appel}.
\end{itemize}
W naszym języku zaimplementowane zostały jedynie typy danych, pattern-matching
oraz wyjątki.

Na tym etapie można przyjąć jedno z kilku ograniczeń tego, co jesteśmy w stanie
zrobić w trakcie działania programu \cite[Chapter 4.1: Data representation]{Appel}:
\begin{enumerate}
\setcounter{enumi}{-1}
\item Nie potrafimy odróżnić wskaźników od liczb całkowitych.
\item Potrafimy odróżnić małe liczby całkowite od wskaźników.
\item Potrafimy odróżnić wszystkie wskaźniki od liczb całkowitych.
\item Potrafimy odróżnić wskaźniki do rekordów o różnej długości.
\end{enumerate}
Od tego, jakie założenia przyjmiemy zależy ile dodatkowych informacji będziemy
musieli przechować w pamięci (np dodając dodatkowy znacznik dla każdej wartości,
mówiącej czy jest ona liczbą całkowitą, czy wskaźnikiem -- pozwoli to pracować
przy założeniu 2, gdy w rzeczywistości maszyna operuje na 0).

W naszym kompilatorze (na tym etapie) przyjęliśmy założenie 1 -- jest ono realne,
gdyż w rzeczywistości wszystkie ,,niskie'' wskaźniki zwykle są specjalnie
pozostawiane wolne, by łatwiej znajdować błędy w programach.
Przy takim założeniu typy danych można podzielić na kilka kategorii, zależnie od
tego, jakie informacje musimy o jego konstruktorach przechować:

\begin{enumerate}
\item ,,Tagged'' -- wymagające przechowania numeru konstruktora i wskaźnika na
  wartość zapisaną w konstruktorze (najogólniejszy),
\item ,,Constant'' -- potrzeba tylko numeru konstruktora, żaden nie posiada
  dodatkowej wartości,
\item ,,Transparent'' -- typ danych ma tylko jeden konstruktor, nie potrzeba żadnej
  informacji zapisywać,
\item ,,TransB'' -- typ danych posiada jeden konstruktor z wartością, wszystkie
  poza nim nie przechowują wartości. Dzięki założeniu 1 możemy użyć jednej komórki
  pamięci.
\end{enumerate}

Pozbycie się konstruktorów wymaga więc sprawdzenia, do jakiej kategorii należy
jego typ danych, a następnie czasem wygenerowania kodu tworzącego rekord (jedno, lub
dwuelementowy) o odpowiednich wartościach (stała będąca numerem konstruktora
i/lub wartość przechowywana w konstruktorze).

Wprowadzenie nowego typu danych nie generuje żadnego kodu.

Dekonstrukcja instancji typu danych (\textit{pattern matching}) wymaga wygenerowania kodu
zależnie od kategorii tego typu danych -- żeby dowiedzieć się, z jakim
konstruktorem mamy do czynienia może być potrzebne wykonanie kilku różnych
czynności:
\begin{itemize}
\item odczytanie jednego z jego pól i porównanie z numerami konstruktorów tego
  typu danych,
\item sprawdzenie czy jest to wskaźnik,
\item czasem też nie trzeba nic robić.
\end{itemize}

Sytuacja z wyjątkami jest jeszcze gorsza -- wyjątki różnią się od zwykłych
typów danych tym, że na etapie kompilacji danej jednostki nie jest wiadome ile
będzie typów
konstruktorów. Dzieje się tak, ponieważ moduły są kompilowane oddzielnie, i
każdy może zdefiniować swoje wyjątki. Z tego powodu wyjątki są reprezentowane
jako rekordy dwuelementowe, gdzie pierwszy element to (wskaźnik na) wartość
przechowywaną w konstruktorze, a drugi to wskaźnik na ciąg znaków będący nazwą
wyjątku. Rzucenie wyjątku to wówczas zapisanie takiej wartości w odpowiednie
miejsce, a złapanie to porównanie wskaźnika na nazwę ze znanymi wartościami.
Jest to wydajne, ponieważ musimy jedynie porównać wskaźniki (wiemy że wszystkie
instancje wyjątku danego typu będą miały ten sam wskaźnik na nazwę, bo jest on
generowany przez kod w miejscu deklaracji wyjątku).


\section{Continuation-Passing Style (CPS)}
CPS to styl zapisu programu, w którym kontynuacja (sterowanie, kod który ma użyć
zwracanej wartości funkcji) jest przekazywana jawnie jako parametr każdej
funkcji -- zwykle ostatni.

Zakładając, że \lstinline|*&|, \lstinline|+&|, \lstinline|sqrt&| to  odpowiedniki funkcji \lstinline|*|, \lstinline|+|, \lstinline|sqrt|  w CPS, używane
w notacji prefiksowej, to program po transformacji do CPS-u wygląda następująco:

Typowy Program
\begin{lstlisting}
let pyth x y =
 sqrt(x*x + y*y)
\end{lstlisting}

Ten sam program w CPS
\begin{lstlisting}
let pyth& x y k =
 (*& x x (fn (x2) =>
          (*& y y (fn (y2) =>
                   (+& x2 y2 (fn (x2py2) =>
                              (sqrt& x2py2 k))))))

\end{lstlisting}


Jak widać, zapisanie programu w CPS wymaga nazwania każdej wartości pośredniej.
Dzięki temu analiza ile wartości jest żywych (ile rejestrów jest potrzebnych do
przechowywania wartości w danym momencie obliczeń) jest trywialna.


Dodatkowymi zaletami CPS-u są:
\begin{itemize}
\item kolejność wykonywania funkcji jest jasno określona i nie zależy od
  języka zapisu programu -- metajęzyka (może on być ewaluowany zarówno strict, jak i lazy -- te same obliczenia
  zostaną wykonane w tej samej kolejności).
\item łatwa reprezentacja domknięcia funkcji -- w innych językach pośrednich
  często nie pozwala się na zagnieżdżanie funkcji, co wymusza pozbycie się ich z
  języka we wcześniejszych stadiach -- co może utrudnić wiele
  optymalizacji~\cite[Chapter 1.2: Advantages of CPS]{Appel}.
\end{itemize}
Wszystko to sprawia, że CPS jest zapisem doskonałym do optymalizacji
(porównywalnym, o ile nie lepszym, niż SSA wykorzystywane przez LLVM i Clang).



\section{Transformacja do CPS-u}
Przekształcenie języka Lambda do CPS-u można traktować jako funkcję $F$,
przyjmującą jako argumenty $A_L$ -- wyrażenie w języku Lambda oraz $c$ -- kontynuację, a
zwracającą $A_C$ -- wyrażenie w CPS.\@ Transformację rozpoczyna wywołanie $F$ z
wyrażeniem reprezentującym pełen program oraz kontynuacją która zakończy
działanie programu gdy zostanie wywołana.


Całość transformacji została opisana szczegółowo w \cite[Conversion into CPS]{Appel}. Na szczególną uwagę
zasługują pewne szczególne przypadki.

\subsection{Operatory logiczne}
Operatory logiczne w językach z rodziny ML zwracają wartości: \textit{prawdę} lub \textit{fałsz}
(\textit{true} / \textit{false}). W CPS wybierają jedną z dwóch kontynuacji. Aby temu zaradzić
wywołujemy tę samą kontynuację w każdym z dwóch przypadków, ale przekazujemy
różne wartości:
\begin{lstlisting}
F(App (Primop i, E), c) =
   F(E, fun v . PRIMOP(i, [v], [], [c(INT 0), c(INT 1)]))
\end{lstlisting}

Niestety kopiując kontynuację, zwiększamy rozmiar kodu wynikowego -- czasem nawet
wykładniczo! Możemy tego uniknąć używając pomocniczej funkcji, która przechowa
kontynuację:
\begin{lstlisting}
F(App (Primop i, E), c) =
   F(E, fun v . FIX([(k, [x], c(VAR x))],
     PRIMOP(i, [v], [], [APP(VAR k, INT 0), APP(VAR k, INT 1)]))
\end{lstlisting}

\subsection{Operowanie kontynuacjami w języku Lambda}
Operatory \textit{callcc} i \textit{throw} (pozwalające używać mechanizmu kontynuacji bezpośrednio
w języku Lambda) wydają się idealnie pasować do CPS-u:
\begin{lstlisting}
F(App (Primop throw, E), c) =
   F(E, fun v . FIX([(f, [x, j], APP(v, [VAR x]))], c(VAR f))
\end{lstlisting}
\lstinline|throw f| obliczy się do funkcji, która zignoruje swoją kontynuację i zamiast niej
użyje \lstinline|f|.

\begin{lstlisting}
F(App (Primop callcc, E), c) =
  FIX([(k, [x], c(VAR x))], F(E, fun v . APP(v, [VAR k, VAR k])))
\end{lstlisting}
\lstinline|callcc f| zapisuje aktualną kontynuację \lstinline|c| jako funkcję \lstinline|k : fun v . c(v)| i używa
funkcji \lstinline|k| zarówno jako argumentu, jak i kontynuacji \lstinline|f|. Dzięki temu niezależnie
od tego, czy \lstinline|f| zakończy działanie normalnie, czy wywoła swój
argument, sterowanie zostanie przekazane zadanej kontynuacji \lstinline|c|.


\subsection{Wyjątki}
Za wyjątki w języku  Lambda odpowiadają instrukcje \textit{Handle} i \textit{Raise}. W CPS w tym
celu używane są operatory \textit{gethdlr} i \textit{sethdlr}, odpowiadające za trzymanie
wskaźnika do funkcji, która aktualnie obsługuje wyjątki:

\textit{Raise} po prostu wywołuje aktualną funkcję obsługującą wyjątki.
\begin{lstlisting}
F(Raise E, c) = F(E, fun v . PRIMOP(gethdlr, [], [h], [APP(VAR h, [v])]))
\end{lstlisting}

\textit{Handle} natomiast musi ją odpowiednio ustawić, i na koniec przywrócić:
\begin{lstlisting}
F(Handle(A, B), c) =
  PRIMOP(gethdlr, [], [h],
    [FIX([(k, [x], c(VAR x)),
            (n, [e], PRIMOP(sethdlr, [VAR h], [],
                         [F(B, fun f . APP(f, [VAR e, VAR k]))]))],
      PRIMOP(sethdlr, [VAR n], [],
        [F(A, fun v . PRIMOP(sethdlr, [VAR h], [] [APP(VAR k, [v])]))]))])
\end{lstlisting}

Jednakże wzajemne zachowanie \textit{callcc} i \textit{Handle} prowadzi do problemu: w aktualnym rozwiązaniu
kontynuacja jest wywoływana z funkcją obsługującą wyjątki wiązaną dynamicznie
(taką, jak zostawił wywołujący), natomiast powinna być ona wiązana statycznie
(taka, jaką miała funkcja tworząca kontynuację). Z tego powodu \textit{callcc} musi
odczytać i przywrócić odpowiednią funkcję obsługującą wyjątki:
\begin{lstlisting}
F(App (Primop callcc, E), c) =
  PRIMOP(gethdlr, [], [h],
    [FIX([(k, [x], c(VAR x)),
          (k', [x'], 
              PRIMOP(sethdlr, [VAR h], [], [APP(VAR k, [VAR x'])]))],
      F(E, fun v . APP(v, [VAR k’, VAR k]))])
\end{lstlisting}

\subsection{Case}
Dodatkowo, instrukcja \textit{Case} (ma ona
wybrać odpowiedni przypadek ze skończonej listy) może być przekształcona na
wiele sposobów  -- \cite[Chapter 5.7: Case statements]{Appel} sugeruje kilka
możliwości optymalizacji: binarne przeszukiwania, drzewa decyzyjne, liniowe
porównania, tablicę skoków. W naszej implementacji wykorzystaliśmy najprostszą
do napisania: liniowe przeszukiwanie. W większości przypadków lista możliwości
będzie bardzo krótka -- praktycznie jedynie generatory kodu tworzą więcej niż
kilka przypadków.




\section{Optymalizacje kodu w CPS-ie}
Dotychczas celem naszej kompilacji było uzyskanie kodu w CPS-ie. Stało się tak,
ponieważ uznaliśmy CPS za dobry język do optymalizacji. Zaimplementowaliśmy
następujące:
\begin{itemize}
\item ,,proste'' \begin{itemize}
  \item zwijanie stałych,
  \item $\eta$-redukcja,
  \item uncurrying,
  \end{itemize}
\item ,,zaawansowane'' \begin{itemize}
  \item $\beta$-ekspansja,
  \item hoisting (zmiana kolejności wyrażeń),
  \item eliminacja wspólnych podwyrażeń.
  \end{itemize}
\end{itemize}

Najwięcej problemów
pojawiło się z zaawansowanymi optymalizacjami.

\subsection{$\beta$-ekspansja}
\textit{$\beta$-ekspansja} (wielokrotna $\beta$-redukcja) to zamiana wywołania funkcji na ciało funkcji z
podstawionymi argumentami. Jest to przydatna optymalizacja (może pozwolić na
działanie wielu innymi), ale niebezpieczna (powiększa rozmiar programu, bez
gwarancji na poprawę). Niestety nie da się z pewnością stwierdzić, kiedy ta
optymalizacja coś da \cite[Chapter 7.1: When to do in-line expansion]{Appel}. Dlatego stosuje się heurystyki, określające czy
warto. Starają się one ocenić proste kryterium: czy po zastosowaniu $\beta$-ekspansji
i kolejnych optymalizacjach rozmiar kodu się zmniejszy. Niestety ponownie nie
jest to problem, który da się policzyć z całą pewnością inaczej niż wykonując te
działania. Nasza ocena (wzorowana na \cite[Chapter 7.2: Estimating the
savings]{Appel}, \cite[Chapter 7.3: Runaway Expansion]{Appel}) opiera się na przypisaniu
każdej funkcji pewnej oceny jej rozmiaru i porównywaniu szacowanych zmian w
rozmiarze programu z pewną stałą, pomniejszoną o głębokość (aby uniknąć
rozwijania ciała już rozwiniętej funkcji) oraz numer rundy (aby przypadkiem nie
pozostać w cyklu, np w funkcjach rekurencyjnych). Używając takich heurystyk
wykonujemy na zmianę $\beta$-ekspansję oraz  ,,proste'' optymalizacje
aż osiągniemy punkt stały.


\subsection{Hoisting}
\textit{Hoisting} polega na zmianie kolejności wyrażeń. W szczególności może to pozwolić
na połączenie definicji funkcji, lub wyjęcie pewnego wyrażenia z pętli, lub
niewykonanie potencjalnie drogiej operacji jeśli jej wynik jest potrzebny tylko
w jednej gałęzi instrukcji warunkowej. Definicja, w jakim przypadku
optymalizacja ta może być wykonana nie jest ciężka, ale żmudna \cite[Chapter
8.2: Rules for hoisting]{Appel}. Wybór
czy należy ją wykonać, podobnie jak w wypadku $\beta$-ekspansji, nie jest
trywialny. Appel \cite[Chapter 8.3: Hoisting optimizations]{Appel} sugeruje
zachłanne podejście: zamiana kolejności tylko jeśli mamy gwarancję zmniejszenia
czasu wykonania programu. To oznacza, m.in. że nigdy nie wyjmiemy wyrażenia z pętli
(gdyż nie mamy gwarancji że wykona się ona przynajmniej raz). Definicje funkcji
będziemy najpierw spychali w dół, a jeśli to się nie uda, to w górę. W ten
sposób połączenie dwóch funkcji zawsze zmniejsza czas wykonania.


\subsection{Eliminacja wspólnych podwyrażeń}
Eliminację wspólnych podwyrażeń (\textit{Common Subexpression Elimination} --
CSE) najprościej jest pokazać na przykładzie \cite[Chapter 9: Common subexpressions]{Appel}:
\begin{lstlisting}
val z = a*b*c+a*b*c
\end{lstlisting}

W CPS będzie on wyglądać następująco:
\begin{lstlisting}
PRIMOP(*,[VAR a, VAR b], [u], [
  PRIMOP(*, [VAR u, VAR c], [v], [
    PRIMOP(*, [VAR a, VAR b], [w], [
      PRIMOP(*, [VAR w, VAR c], [x], [
        PRIMOP(+, [VAR v, VAR x], [z], [ ...
\end{lstlisting}

Wartość \lstinline|a*b| jest liczona podwójnie. Możemy więc zastąpić \lstinline|w| przez \lstinline|u|:
\begin{lstlisting}
PRIMOP(*,[VAR a, VAR b], [u], [
  PRIMOP(*, [VAR u, VAR c], [v], [
      PRIMOP(*, [VAR u, VAR c], [x], [
        PRIMOP(+, [VAR v, VAR x], [z], [ ...
\end{lstlisting}

To z kolei pozwala zauważyć, że wartość \lstinline|u*c| jest liczona podwójnie. Zastępujemy więc
\lstinline|x| przez \lstinline|v|:
\begin{lstlisting}
PRIMOP(*,[VAR a, VAR b], [u], [
  PRIMOP(*, [VAR u, VAR c], [v], [
        PRIMOP(+, [VAR v, VAR v], [z], [ ...
\end{lstlisting}
To z kolei odpowiada wyrażeniu
\begin{lstlisting}
val z = let val v = a*b*c in v+v
\end{lstlisting}


\section{Droga do kodu maszynowego}
Po otrzymaniu zoptymalizowanego kodu w CPS-ie pozostaje tylko przepisać go
na wykonywalny na danej maszynie. Problemy na jakie napotkaliśmy to:
\begin{itemize}
  \item zagnieżdżone definicje funkcji -- wywołanie funkcji w architekturze von
    Neumanna jest proste, problem pojawia się gdy funkcja posiada zmienne wolne,
\item ilość żywych (potrzebnych w danym momencie obliczeń) wartości może
  przewyższyć ilość rejestrów maszyny na którą generujemy kod,
\item brakuje nam automatycznego zarządzania pamięcią.
\end{itemize}


\subsection{Domknięcia}
Za rozwiązanie pierwszego problemu odpowiada dodanie domknięć (\textit{closure
conversion}) \cite[Chapter 10: Closure conversion]{Appel}. Polega ono na prostej zmianie: każda funkcja (zarówno przekazywana
jawnie, jak i kontynuacja) przestaje być ,,tylko'' prostym wskaźnikiem na funkcję,
a staje się strukturą danych o znanej strukturze, która przechowuje zarówno
adres funkcji, jak i jej zmienne wolne. Struktura ta jest bardzo prosta -- jest
to rekord, którego pierwszym elementem jest wskaźnik na funkcję, a kolejnymi
jego elementami są zmienne wolne.

Wywołanie funkcji wiąże się ze stworzeniem takiej struktury. Z tego powodu
przydatne jest znalezienie takich funkcji które nie potrzebują tej dodatkowej
struktury. Nazywamy takie funkcje \textit{znanymi} -- są to funkcje, które nie są nigdy
przekazywane jako argument. Dzięki temu znamy wszystkie miejsca gdzie dana
funkcja jest wywoływana i możemy ją przekształcić -- przekazując wszystkie jej
zmienne wolne jako argumenty.

\subsection{Rozlewanie rejestrów}
Aby upewnić się, że ilość potrzebnych żywych zmiennych w danym momencie nie
przekracza ilości rejestrów (a zatem kod daje się wykonać) wykonujemy proces
zwany rozlewaniem rejestrów (\textit{register spilling}) \cite[Register
spilling]{Appel}. Polega on na wykryciu
takich sytuacji w czasie kompilacji i rozwiązaniu problemu przez alokację
dodatkowych rekordów na stosie. Dzięki temu zamiast potrzebować n rejestrów na
zmienne, możemy użyć jednego rejestru wskazującego na rekord n-elementowy.
Oczywiście ceną za takie posunięcie jest konieczność tworzenia i ładowania
wartości do oraz z takiego rekordu w trakcie wykonania programu.


\subsection{Odśmiecanie pamięci}
Do korzystania z automatycznego odśmiecania pamięci (\textit{garbage collection})
potrzebujemy ustalić:
\begin{itemize}
\item  kiedy uruchomić odśmiecanie pamięci, oraz
\item które wartości są osiągalne (a zatem które możemy usunąć).
\end{itemize}
Okazuje się, że odpowiedź na pierwsze pytanie może być prosta: przed wywołaniem
każdej funkcji sprawdzamy, ile wolnej pamięci pozostało, i jeśli jest jej za
mało uruchamiamy procedurę odśmiecania pamięci.

Pozostaje tylko ustalić, skąd odśmiecanie ma wiedzieć które wartości są
osiągalne. W tym celu trzeba przyjąć albo bardzo silne założenia o systemie
zarządzania pamięcią, albo dodać do obiektów pewne metadane o ich zawartości.
Podobnie jak Appel \cite[Chapter 16.4: Runtime data formats]{Appel} założyliśmy:
\begin{itemize}
\item przed każdym punktem wejścia do funkcji znajduje się wskaźnik na jej
  początek,
\item przed początkiem funkcji znajduje się informacja o używanych przez nią
  rejestrach,
\item przed każdym rekordem znajduje się blok metadanych (określający, jego
  długość oraz czy może zawierać wskaźniki -- jeśli tak, to każdy jego element
  jest albo wskaźnikiem,albo liczbą całkowitą ze znacznikiem (\textit{tagged integer}), w przeciwnym wypadku
  może zawierać dowolne dane nie będące wskaźnikami).
\end{itemize}

Zauważmy, że jest to wszystko, co potrzebujemy -- ponieważ nie mamy stosu. Razem
pozwala to zaimplementować szybkie odśmiecanie pamięci przez kopiowanie
generacjami (\textit{generational garbage collection}).

Opiera się ono na dwóch obserwacjach dotyczących języków funkcyjnych:
\begin{enumerate}
\item nowe obiekty są częściej usuwane od starych,
\item nowe obiekty trzymają referencje na starsze.
\end{enumerate}

Wynika to z trwałych struktur danych używanych często w programowaniu
funkcyjnym. W przypadku zagnieżdżania takich struktur konstruktor przyjmuje jako
argument starszy obiekt (a więc musiał on powstać przed tym wywołaniem
konstruktora) oraz trzyma do niego referencję (a więc musi zostać usunięty przed
nim). Jedynym wyjątkiem są modyfikowalne komórki pamięci (o tym za chwilę).

Generacjami będziemy nazywać obszary pamięci, w których będziemy alokować
obiekty. Kiedy \textit{n}-ty obszar się zapełni, przechodzimy do nowego -- \textit{n+1} -- będziemy
nazywać go \textit{generacją n+1}. Numery generacji nigdy się nie zmniejszają. Aby
odzyskać pamięć, użyjemy informacji o aktualnie wykonywanej funkcji, by znaleźć
wszystkie żywe obiekty w najnowszych generacjach (zwykle wystarczą 2-3), a
następnie skopiujemy je w nowy obszar i zwolnimy pamięć ze starego. Dzięki
obserwacji 1. wiemy, że usunęliśmy śmieci z obszarów w których było ich
najwięcej, a dzięki obserwacji 2. wiemy, że nie zepsuliśmy referencji ze starych
generacji. Z wyjątkiem modyfikowalnych komórek pamięci. Problem z nimi  możemy
rozwiązać na kilka sposobów \cite[Chapter 16.3: Generational garbage
collection]{Appel}: poprzez dwupoziomowe wskaźniki (\textit{assignment
table}) \cite{Lieberman83}, trzymanie zbioru komórek które mają wskaźniki do nowszych komórek od
siebie \cite{Shaw87}, lub użycie pamięci wirtualnej \cite{Ungar86}.


\section{Ograniczenia}
Z różnych przyczyn stworzony przez nas kompilator nie osiągnął poziomu
generowania faktycznego kodu maszynowego jednej z aktualnych architektur
procesora. Podczas jego pisania rozważane było użycie języka pośredniego LLVM
jako ostatniej fazy (z niego kod maszynowy na różne architektury potrafi
generować szereg narzędzi składających się na kompilator Clang). Niestety
okazało się, że ten język pośredni nie pasuje do założeń przyjętych w książce
Appela -- pod pewnymi względami jest o wiele silniejszy, pod innymi brakuje mu
pewnych rzeczy. Z tego powodu stwierdziliśmy że bardziej kształcące będzie
wygenerowanie kodu pewnej abstrakcyjnej maszyny. Z tego też powodu nie ma
implementacji zarządzania pamięcią, ani innych zależnych od platformy części.

\section{Podsumowanie}
Okazuje się, że napisanie kompilatora, nawet wysokiego poziomu, nie jest ciężkie
--- duża część teorii jest przystępna i łatwa do implementacji. Z drugiej strony, jest
też duże pole do rozwoju -- pojawiają się problemy nietrywialne, a całość
wymaga dobrej organizacji i staranności. Być może napisany na seminarium
kompilator zostanie w przyszłości rozwinięty o brakujące mu części. Mimo pewnych
braków, uważam że Implementacja Języków Funkcyjnych zakończyła się sukcesem --
zaimplementowaliśmy wszystkie główne części kompilatora.

\bibliographystyle{plain}
\bibliography{books}


\end{document}